\section{Cognition and evaluation}
\label{cognition-and-evaluation}

The `strong' version of the distributional hypothesis holds that the linguistic
contexts in which a word appears have a causal relationship to its cognitive
representation \parencites[16-18]{Lenci2008}[18-19]{Lenci2023}.
This interpretation derives from psychology and cognitive science, where distributional
methods have been widely applied
\parencites[e.g.][16]{Lenci2008}[15-16]{Lenci2023}{Landauer2011}.
Indeed, some of the most influential work in distributional semantics has been
developed and evaluated in the context of cognitive science, and vector representations
of meaning in psychology predate the development of vector-space models
\parencref{meaning-and-use,count-based-models}.
In response to the problem of induction, for instance, \textcites{Landauer1997}
describe Latent Semantic Analysis as a theory of how we learn and use the meanings of
words \parencite[cf.][3-34]{Landauer2011}.
There is likewise a relation between the cognitive interpretation and a `use theory of
meaning' \parencites[22-23]{Lenci2023}.
This view is notably advocated by \textcites{Miller1991}, who describes the `contextual
representation' of a word as ``knowledge of how that word is used''
\parencites*[4]{Miller1991}.
Hence, the `strong' version of their `contextual hypothesis' is that the semantic
similarity of two words is determined by the similarity of their contextual, i.e.,
cognitive, representations \parencites*[8]{Miller1991}.
I take the view that this interpretation of the distributional hypothesis is
unwarranted, and that it is due principally to (a) the practical necessity of
evaluating distributional models against human judgments and (b) the widespread usage
of cognitive metaphors in the literature \parencites[569-570]{Gastaldi2021a}.
I discuss these points in turn, before concluding.

The evaluation of language models may be divided into \emph{extrinsic} evaluation,
which is based on the performance of the model on a specific task, and \emph{intrinsic}
evaluation, which is notionally separate from any such task
\parencites[38]{Wang2020a}{Torregrossa2021}[91-96]{Bommasani2022}.
This distinction is particularly important for the evaluation of foundation models
\parencref{contextual-language-models}, which are explicitly intended to be
task-agnostic.
Intrinsic evaluation is largely based on correspondence with human judgments, e.g.,
ratings of semantic similarity or relatedness \parencites[17]{Lenci2023}, as evidenced
by the survey of evaluation datasets in \textcites[1281]{Lenci2022}.
For example, the word analogy task \parencref{compositionality} has been used to
evaluate the quality of embedding spaces
\parencites[4-5]{Mnih2013}{Levy2014}{Pennington2014}, but the generality of this
approach has been questioned \parencites[1300]{Lenci2022}.
There is likewise cause to criticize word similarity as an evaluation methodology
altogether \parencites[]{Batchkarov2016}, not least due to the incompatibility of a
naïve geometric definition with observable cognitive phenomena
\parencites[3-11]{Lewis2022}.
Performance on an intrinsic word-similarity task does not necessarily translate to
extrinsic downstream tasks \parencites[7-8]{Batchkarov2016}, and inter-annotator
agreement is generally poor for word-similarity in comparison to more specific tasks
\parencites[8-9]{Batchkarov2016}.
The correspondences between intrinsic and extrinsic evaluation are thus an important
topic of research \parencites[92-94]{Bommasani2022}.
While evaluation with respect to human judgments is practically necessary, it implies a
cognitive interpretation of the model in question, which may be misleading.
Accordingly, \textcites{DeDeyne2016} show that models explicitly based on
word-association can outperform DSMs.

The language we use to describe computers and artificial-intelligence systems makes
widespread use of cognitive metaphors: computers and LSTM networks have `memory',
Transformer blocks `pay attention to' textual features, and so on.
This usage largely originates in the connectionist tradition and `neural' language
models \parencref{predictive-language-models}.
Interpreted literally, these metaphors lead us towards the misconception that
increasingly sophisticated artificial-intelligence systems are increasingly akin to the
mind \parencites[103]{Hacker2019}.
The crux of \citeauthor{Hacker2019}'s argument against this view is that the faculties
of the mind are \emph{normative}, whereas the behaviour of machines is \emph{causal}:
following a rule is something other than executing a program, and determining whether
the result of executing a program is correct is something other than verifying its
`causal inevitability' \parencites*[107-108]{Hacker2019}.
We do not speak of less sophisticated calculating devices as `thinking', except
metaphorically, and we must take care to retain this metaphorical distinction when we
speak of artificial-intelligence systems like language models.
Indeed, it is difficult to see how particular DSMs could be interpreted as models of
cognitive linguistic processes, since they commonly employ different architectures
\parencites[570]{Gastaldi2021}.
As \textcites{Houghton2023} have noted, state-of-the-art language models are not
necessarily more like the brain than their predecessors: recurrent neural networks have
generally been superseded by Transformers \parencites{Wang2019b}{Gillioz2020}, which
eschew recurrent connections for the sake of computational efficiency
\parencref{contextual-language-models}, but humans are generally considered to process
language sequentially \parencites{Dominey2003}.
What is common to these models is the theoretical principle of the distributional
hypothesis, and that they learn distributed representations of the `meanings' of
linguistic units (specifically, tensors in a high-dimensional vector space).
Language-model training is furthermore increasingly dissimilar to natural language
acquisition: rather than ingesting vast textual corpora, we use language to
\emph{interact} with each other to achieve shared goals in our environment
\parencites[361-363]{Lenci2023}.
Hence, I suggest, the successes of DSMs in modelling cognitive phenomena vindicate this
theoretical principle and representational form, but not more.

The `structuralist hypothesis' of \textcites{Gastaldi2021}{Gastaldi2021a} holds that
distributional semantics manifests purely formal relations between linguistic units,
which are strictly separate from cognitive representations and processes, and that its
successes are principally due to the reality of this internal structure.
We might expect that this structure bears some resemblance to that of the concepts of
our experience which it signifies -- perhaps even as a consequence of the cognitive
processes that underlie its acquisition and use \parencites[see][]{Goldsmith2005}.
As \textcites[5-8]{Landauer2011} explains, language is in part a map of the world, and
we can derive useful knowledge from the map alone.
But ``the map is not the territory'' \parencites[58]{Korzybski1995}, and the relations
between the map and the territory are perhaps beyond the scope of distributional
semantics.
