\section{Distributional structure}
\label{distributional-structure}

The distributional hypothesis is typically stated in terms such as ``words that occur
in similar contexts tend to have similar meanings'' \parencites[142-143]{Turney2010}
or, more precisely, ``the degree of semantic similarity between two linguistic
expressions $A$ and $B$ is a function of the similarity of the linguistic contexts in
which $A$ and $B$ can appear'' \parencites[3]{Lenci2008}.
Aside from Wittgenstein and Firth \parencites[10-14]{Lenci2023}, the most frequent
shorthand for a distributional approach to semantics in the literature is Zellig
Harris's \citetitle{Harris1954} \parencites*{Harris1954}.
However, Harris was principally concerned with phonology and morphology
\parencites[6]{Lenci2023}, despite the strong association between `distributionalism'
and semantics today \parencites[186]{Gastaldi2021}[579]{Gastaldi2021a}.
This is not to say that distributionalism is not concerned with meaning, but rather
that it is limited to certain of its aspects, which I revisit ahead.
Rather than a general semantics, Harris's aim was to establish a methodological basis
for linguistics as a science \parencites[3,26]{Lenci2008}, which ``should (and, indeed,
could) only deal with what is \emph{internal} to language''
\parencites[3]{Sahlgren2008}.
As \citeauthor{Lenci2023} explain, linguistics is unique in that ``it does not have
recourse to a metalanguage that is external to its object of study''
\parencites*[6]{Lenci2023}.
Harris's ideas are properly situated with respect to his \emph{structuralist}
intellectual heritage: he was a student of Leonard Bloomfield, the pre-eminent American
structural linguist, who was deeply influenced by Ferdinand de~Saussure
\parencites[20-30]{Matthews2001}[572]{Gastaldi2021a}.
We may thus look to Saussure to understand what it means for words to occur in `similar
contexts' and to have `similar meanings'.

The term `structural linguistics' typically refers to an intellectual trend popularized
by the posthumous publication of Saussure's \citetitle{Saussure2011}
\parencites[3-4,6]{Matthews2001}.
In this work, Saussure defines the linguistic sign as an indivisible unit that relates
a concept, which is \emph{signified}, to a sound-image, which is its \emph{signifier}
\parencites*[65-67]{Saussure2011}.
Central to the theory is that the sign is `arbitrary': there is nothing external to the
language-system that determines the sound-image that stands for a given concept
\parencites*[67-70]{Saussure2011}.
Hence, a sign is distinguished solely by its functional differences from the other
signs in the language system, not by the extra-linguistic relations of the concept(s)
that it signifies.
Such a conception of meaning is thus \emph{differential} rather than referential
\parencites[5]{Sahlgren2008}: like Wittgenstein, Saussure rejects the age-old view that
``words are names for `things''' \parencites[18]{Matthews2001}[11]{Lenci2023}.
Saussure further separates the differences between signs into \emph{syntagmatic} and
associative or, following Hjelmslev \parencite*{Hjelmslev1938}, \emph{paradigmatic}
relations.
In distributional semantics, these kinds of relations between words are commonly known
as first-order and second-order co-occurrence \parencites[126]{Jurafsky2023}, or ``how
typical they are as neighbors and how well they are substitutable for each other''
\parencites[1;~cf.~\cref{contextual-language-models}]{Schutze1993}.

The geometric definition of semantic similarity induced by vector-space models, whose
representations of `context' have been associated with syntagmatic and paradigmatic
relations (\cite{Sahlgren2008}; cf. \cref{count-based-models}), is sometimes criticized
due to its breadth \parencites[2]{Pado2003}.
The literature thus commonly distinguishes between semantic similarity and a broader
notion of word \emph{relatedness} \parencites{Budanitsky2006}[105-106]{Jurafsky2023}; a
thorough analysis of similarity measures is provided by \textcites{Curran2004}.
It is well-established that the neighbours of distributional representations reflect a
mixture of semantic relations, e.g., synonymy and antonymy.
However, as \textcites[4]{Sahlgren2008} notes, for this behaviour to problematize the
distributional definition, one must suppose a prescriptivist view wherein these
semantic relations are given \emph{a~priori} \parencite[188]{Gastaldi2021}.
In any case, an appeal to Harris's distributionalism suggests that `meaning', insofar
as it is amenable to linguistic analysis, derives from the formal relations between
signs, rather than from the material relations between referents in the world, or
contextually-determined cognitive representations.
In the following sections, I examine the mechanics of distributional models to assess
whether and how they actualize such a conception of meaning.
